\begin{challenge}
    \chatitle{Using the terminal-emulator to navigate the linux filesystem}
    \begin{chadescription}
    As you should know by now, we have to differentiate between `memory` and `storage`.
    If you are not sure about the differences, please go back to the \href{https://www.github.com/STEMgraph/}{first Filesystem Challenge}.
    We interface with our filesystem, using the operating systems kernel. 
    It talks to the filesystem driver and the hardware drivers, to perform operations on the filesystem.
    The major operation we will be working with here is the \texttt{cd} command.
    A general rule of thumb in user-interface design is to make commonly used commands as short as possible.
    In the first Filesystem Challenge you already learned about filenames and their connection to the harddrive. 
    You should remember, that there are no such things as directories or folders on your actual harddrive.
    These are just inventions by the filesystem. 
    We commonly use pre- and suffixes to tag different files on our computer. 
    While wie use suffixes to indicate the functional content of a file, such as \texttt{pdf}, \texttt{docx}, \texttt{jpg}, we use prefixes to indicate the context of the file, such as \texttt{Documents}, \texttt{Downloads} and so on.
    Hence a directory is just a collection of files with the same prefix.
    Coming back to the idea of making commonly used commands as short as possible, we are going to focus on the \texttt{cd} command.
    Modern shells, or terminal interpreters, help us handling long filenames with a lot of prefixes by giving us the impression, that there are no other files on our filesystem, but only files with the same prefix.
    This is the actual meaning of being in a directory.
    It makes all files with the same prefix easier to use. 
    A usual command line prompt looks like this:
    \begin{center}
    \texttt{user@host:/home/user\$}
    \end{center}
    The \texttt{user} is the username, \texttt{host} is the hostname, and \texttt{/home/user/} is the current directory.
    Seeing the current directory in the promt is really helpful. 
    It makes it easy to remember where you are.
    But it also indicates, that whatever filename you are now using as an input to a command, will get appended to the current directory.
    In this challenge we will: 
    
    
    \begin{enumerate}
        \item use the \texttt{cd} command to navigate your filesystem
        \item figure out which files are in your current directory by running \texttt{ls} in your terminal
        \item let the terminal print the current directories name by using the \texttt{pwd} command
    \end{enumerate}
    \end{chadescription}

    \begin{task}
    Open your terminal-emulator on your linux system.
        \begin{questions}
            \item Type \texttt{pwd} and press \texttt{Enter}. What is the output?
        \end{questions}
    You will most likely see, that the output is not what your promt shows you. 
    It's unfortunate, that we are starting with a special case here.
    When you fire up your terminal, the present working directory is set to your home directory.
    The \texttt{~} symbol is a special symbol in the promt, that indicates your home directory.
    This symbol is the same for all users. 
    But of course it does not refer to the same directory for every user, but to their very own home directory.
    Your own home directory of course has a regular name, such as \texttt{/home/user}.
    This is why the output of \texttt{pwd} is \texttt{/home/user} and not \texttt{~}.
    \end{task}

    \begin{task}
    We will now change the prefix of our current directory.
    The command for this action, which indicates to the shell, that we want to change the current directory, is the \texttt{cd} command.
    In the most usual case it takes one argument, which is the name of the new directory.
        \begin{questions}
            \item Which command can you use to start \texttt{xeyes}?
            \item Explain in your own words, what happens when a user-program wants to display something using the X11 protocol!
        \end{questions}
    If this didn't work as expected for you, go back to the \href{https://www.github.com/STEMgraph/}{first X-Server Challenge}.
    \end{task}
    \begin{task}
    Use your preferred package manager to install the \texttt{xpdf} package. If you want to build xpdf from source, follow the Link to the \href{https://gitlab.com/xpdf-mirror/xpdf}{xpdf} repository.
        \begin{questions}
            \item Which command can you use to install the \texttt{xpdf} package?
            \item In which programming language was xpdf written?
            \item Can you figure out, which GUI Library xpdf uses?
            \item Run \texttt{xpdf --version}, what is the output?
        \end{questions}
    \end{task}
    \begin{task}
    \texttt{xpdf} can display PDF files. Run \texttt{xpdf <filename>} to display the PDF file \textit{<filename>}. If you don't have a PDF file, you can download a simple one from \href{https://constitutioncenter.org/media/files/constitution.pdf}{https://constitutioncenter.org/media/files/constitution.pdf}. To download a file into your \texttt{\textasciitilde} directory, use \texttt{wget <URL>}.
        \begin{questions}
            \item Run the command \texttt{xpdf -z 8}, what is the output?
            \item Run the command \texttt{xpdf -z 50}; then run the command \texttt{xpdf -z 150}, what is the output?
            \item Run the command \texttt{xpdf -rv}, what is the output?
        \end{questions}
    \end{task}
    \begin{task}
    While installing xpdf, you also installed the \texttt{poppler-utils} package. Part of this package is the \texttt{pdftotext} command.
        \begin{questions}
            \item What does the command \texttt{pdftotext <filename>} do?
        \end{questions}
    \end{task}
\end{challenge}
    
\learningobjective{At the end of this challenge, the scholar will be able to use the cd-, ls- and pwd-command to traverse their POSIX filesystem.}
\begin{challenge}
    \chatitle{Using the terminal-emulator to navigate the linux filesystem}
    \begin{chadescription}
    As you should know by now, we have to differentiate between `memory` and `storage`.
    If you are not sure about the differences, please go back to the \href{https://www.github.com/STEMgraph/}{first Filesystem Challenge}.
    We interface with our filesystem, using the operating systems kernel. 
    It talks to the filesystem driver and the hardware drivers, to perform operations on the filesystem.
    The major operation we will be working with here is the \texttt{cd} command.
    A general rule of thumb in user-interface design is to make commonly used commands as short as possible.
    In the first Filesystem Challenge you already learned about filenames and their connection to the harddrive. 
    You should remember, that there are no such things as directories or folders on your actual harddrive.
    These are just inventions by the filesystem. 
    We commonly use pre- and suffixes to tag different files on our computer. 
    While we use suffixes to indicate the functional content of a file, such as \texttt{pdf}, \texttt{docx}, \texttt{jpg}, we use prefixes to indicate the context of the file, such as \texttt{Documents}, \texttt{Downloads} and so on.
    Hence a directory is just a collection of files with the same prefix.
    Coming back to the idea of making commonly used commands as short as possible, we are going to focus on the \texttt{cd} command.
    Modern shells, or terminal interpreters, help us handling long filenames with a lot of prefixes by giving us the impression, that there are no other files on our filesystem, but only files with the same prefix.
    This is the actual meaning of being in a directory.
    It makes all files with the same prefix easier to use. 
    A usual command line prompt looks like this:
    \begin{center}
    \texttt{user@host:/home/user\$}
    \end{center}
    The \texttt{user} is the username, \texttt{host} is the hostname, and \texttt{/home/user/} is the current directory.
    Seeing the current directory in the promt is really helpful. 
    It makes it easy to remember where you are.
    But it also indicates, that whatever filename you are now using as an input to a command, will get appended to the current directory.
    In this challenge we will: 
    
    
    \begin{enumerate}
        \item use the \texttt{cd} command to navigate your filesystem
        \item figure out which files are in your current directory by running \texttt{ls} in your terminal
        \item let the terminal print the current directories name by using the \texttt{pwd} command
    \end{enumerate}
    \end{chadescription}

    \begin{task}
    Open your terminal-emulator on your linux system.
        \begin{questions}
            \item Type \texttt{pwd} and press \texttt{Enter}. What is the output?
        \end{questions}
    You will most likely see, that the output is not what your promt shows you. 
    It's unfortunate, that we are starting with a special case here.
    When you fire up your terminal, the present working directory is set to your home directory.
    The \texttt{~} symbol is a special symbol in the promt, that indicates your home directory.
    This symbol is the same for all users. 
    But of course it does not refer to the same directory for every user, but to their very own home directory.
    Your own home directory of course has a regular name, such as \texttt{/home/<username>}.
    This is why the output of \texttt{pwd} is \texttt{/home/<username>} and not \texttt{~}.
    \end{task}

    \begin{task}
        We will now use the \texttt{ls} command to list all files in our present working directory.
        \begin{questions}
            \item Type \texttt{ls} and press \texttt{Enter}. What is the output?
        \end{questions}
    \end{task}

    \begin{task}
        As you may have noticed, the output might be empty.
        Especially if this is the first time you work with your filesystem.
        Therefor we will now see what's inside of some other directories. 
        \begin{questions}
            \item Let's research the content of the root-directory. Type \texttt{ls /} and press \texttt{Enter}. What is the output?
            \item Let's also see what's inside of the \texttt{home} directory. Type \texttt{ls /home} and press \texttt{Enter}. What is the output?
        \end{questions}
        In the first case, you will see the most basic directory of your filesystem.
        All of the other directories will be inside of this one.
        In the second case, you will see the directories of the home directory.
        All users have their own home directory, which are usually located in the \texttt{/home} directory.
    \end{task}

    \begin{task}
        We may also use additional options for the \texttt{ls} command.
        Let's try some useful ones and you try to explain what these options change in the output!
        \begin{questions}
            \item Type \texttt{ls -l} and press \texttt{Enter}. What is the output?
            \item Type \texttt{ls -lS} and press \texttt{Enter}. What is the output?
            \item Type \texttt{ls -lh} and press \texttt{Enter}. What is the output?
            \item Type \texttt{ls -la} and press \texttt{Enter}. What is the output?
            \item Type \texttt{ls -l /} and press \texttt{Enter}. What is the output?
            \item Type \texttt{ls -l /home} and press \texttt{Enter}. What is the output?
        \end{questions}
    \end{task}

    \begin{task}
    We will now change the prefix of our current directory.
    The command for this action, which indicates to the shell, that we want to change the current directory, is the \texttt{cd} command.
    In the most usual case it takes one argument, which is the name of the new directory.
        \begin{questions}
            \item First type the command \texttt{pwd} and remember the output. 
            \item Now type \texttt{cd /} and press \texttt{Enter}. Now run \texttt{ls -l}. What is the output?
            \item Run \texttt{cd <the path you saved two steps earlier>} and press \texttt{Enter}. Run \texttt{pwd} again. Where did you end up?
            \item Go back to the -so called- root-directory, or \texttt{/}. Run \texttt{ls -l} again. Choose one of the directories in your filesystem and type \texttt{cd <directory>}. Now run \texttt{ls -l}. What is the output?
        \end{questions}
    \end{task}

    \begin{task}
    While we are navigating the file system, we sometimes just want to go back to specific directories.
    There are three very special commands, that can be used to do this.
        \begin{questions}
            \item Type \texttt{cd}, so without any parameter and press \texttt{Enter}. Run \texttt{pwd}, where did you go?
            \item No type \texttt{cd -}, run \texttt{pwd} again, where did you go now?
            \item Run \texttt{cd ..}, run \texttt{pwd} again, what does this command do?
        \end{questions}
    \end{task}
    \begin{advise}
        If you get tired of typing the complete path when changing a directory, try tapping the \texttt{tab} key.
        It will autocomplete the path for you if possible.
        Are multiple options available and \texttt{tab} can't complete, tap the \texttt{tab} key again for a list of all possibilities!
    \end{advise}
\end{challenge}
    

\learningobjective{At the end of this challenge, the scholar will be able to use the cd-, ls- and pwd-command to traverse their POSIX filesystem.}
\begin{challenge}
    \chatitle{Using the terminal-emulator to explore the *nix filesystem}
    \begin{chadescription}
    We interface with our filesystem, using the operating systems kernel. 
    It talks to the filesystem driver and can store and retrieve data from locations, that are identified by filenames.
    Filenames identify a specific location on the hardware, that would be hard to address otherwise.
    When the operating system wants to retrieve data, it needs to know the hardware-coordinates within the computer's memory or storage.
    The filesystem is the place where the connection between name and physical address is stored. 
    Much like a phonebook, it maps names to physical addresses.\\
    If you have worked with computers before, you probably have heard about directories.
    I am sorry to tell you at this point, that directories are a lie. 
    They are just very long filenames, that are used to identify a collection of files.
    You should always remember, that there are no such things as directories or folders on your actual harddrive.
    These are just inventions by the filesystem. 
    We commonly use pre- and suffixes to tag different files on our computer. 
    While we use suffixes to indicate the functional content of a file, such as \texttt{pdf}, \texttt{docx}, \texttt{jpg}, we use prefixes to indicate the context of the file, such as \texttt{Documents}, \texttt{Downloads} and so on.
    Hence what we think of as a directory is just a collection of files with the same prefix, that get presented to us in a certain way.
    Since we want to make commonly used commands as short as possible, we are going to focus on the \textit{change-directory}-command, or \texttt{cd}.
    Modern shells, or terminal interpreters, help us handling long filenames with a lot of prefixes by giving us the impression, that there are no other files on our filesystem, but only files with the same prefix.
    This is the actual meaning of being in a directory.
    It makes all files with the same prefix easier to use. 
    A usual command line prompt looks like this:
    \begin{center}
    \texttt{user@host:/home/user\$}
    \end{center}
    The \texttt{user} is the username, \texttt{host} is the computers given-name, and \texttt{/home/user/} is the current directory.
    Seeing the current directory in the promt is really helpful. 
    It makes it easy to remember where you are.
    But it also indicates, that whatever filename you are now using as an input to a command, will get appended to the current directory.
    In this challenge we will: 
    
    \begin{enumerate}
        \item use the \texttt{cd} command to navigate your filesystem
        \item figure out which files are in your current directory by running \texttt{ls} in your terminal
        \item let the terminal print the current directories name by using the \texttt{pwd} command
    \end{enumerate}
    \end{chadescription}

    \begin{task}
    Open your terminal-emulator on your linux system.
    Type the command \texttt{pwd} and press \texttt{Enter}.
    The output of the command will be printed to your terminal.
    \texttt{pwd} is short for \texttt{print working directory}.
    Working directory only means, that the shell will prepend the current directory to the input of any command - except for a some special cases we are about to explore.
    You will most likely see, that the output is not what your prompt shows you. 
    It's unfortunate, that we are starting with a special case here.
    When you fire up your terminal, the present working directory is usually set to your \textit{home directory}.
    The \texttt{~} symbol is a special symbol in the prompt, that indicates your home directory.
    This symbol is the same for all users. 
    But of course it does not refer to the same directory for every user, but to their very own home directory.
    Your own home directory of course has a regular name, such as \texttt{/home/<username>}.
    This is why the output of \texttt{pwd} is \texttt{/home/<username>} and not \texttt{~}.
    \begin{questions}
        \item What is the difference between a files prefix and a files suffix? What does the command \texttt{pwd} have to do with that?
    \end{questions}
    \end{task}
    \begin{advice}
        Linguistically, "to prompt" means to incite, suggest, or encourage someone to do something, which is precisely what the CLI prompt does — it signals that the system is ready and waiting for your next instruction.
    \end{advice}

    \begin{task}
        Generate a new file in your home directory by running \texttt{touch <filename>} and press \texttt{Enter}.
        Touch is short for \texttt{touch file}. 
        It's original meaning is to update the timestamp of a files latest modification.
        If you \texttt{touch} with a filename that doesn't exist yet, it will be created without any contents.
        Use the \texttt{ls} command to list all files in our present working directory.
        Take one of the filenames and type \texttt{realpath <filename>} and press \texttt{Enter}.
        \begin{questions}
            \item Why does it show a different name, than just the \texttt{filename.suffix}?
        \end{questions}
    \end{task}

    \begin{task}
        We have now seen all the files, that share our present working directory as a prefix.
        Especially if this is the first time you work with your filesystem, your home-directory might be nealy empty.
        Therefor we will now see what's inside of some other directories. 
        \begin{questions}
            \item Let's research the content of the root-directory. Type \texttt{ls /} and press \texttt{Enter}. What is the output?
            \item Let's also see what's inside of the \texttt{home} directory. Type \texttt{ls /home} and press \texttt{Enter}. What is the output?
        \end{questions}
        In the first case, you will see the most basic directory of your filesystem.
        All of the other directories will be inside of this one.
        If you address any file in your filesystem with a prepended \texttt{/}, the path that you are providing will not be relative to your present working directory. 
        It will be relative to the root directory - also called absolute path.
        Remember: If you give your command-line-interpreter a filepath, it will search for the file in your present working directory, if you prepend a \texttt{/}, it will search in the root directory.
        In the second case, you will see the directories of the home directory.
        All users have their own home directory, which are usually located in the \texttt{/home} directory.
    \end{task}

    \begin{task}
        We may also use additional options for the \texttt{ls} command.
        Let's try some useful ones and you try to explain what these options change in the output!
        \begin{questions}
            \item Type \texttt{ls -l} and press \texttt{Enter}. What is the output?
            \item Type \texttt{ls -lS} and press \texttt{Enter}. What is the output?
            \item Type \texttt{ls -lh} and press \texttt{Enter}. What is the output?
            \item Type \texttt{ls -la} and press \texttt{Enter}. What is the output?
            \item Type \texttt{ls -l /} and press \texttt{Enter}. What is the output?
            \item Type \texttt{ls -l /home} and press \texttt{Enter}. What is the output?
        \end{questions}
    \end{task}

    \begin{task}
    We will now change the prefix of our current directory.
    The command for this action, which indicates to the shell, that we want to change the current directory, is the \texttt{cd} command.
    In the most usual case it takes one argument, which is the name of the new directory.
        \begin{questions}
            \item First type the command \texttt{pwd} and remember the output. 
            \item Now type \texttt{cd /} and press \texttt{Enter}. Now run \texttt{ls -l}. What is the output?
            \item Run \texttt{cd <the path you saved two steps earlier>} and press \texttt{Enter}. Run \texttt{pwd} again. Where did you end up?
            \item Go back to the -so called- root-directory, or \texttt{/}. Run \texttt{ls -l} again. Choose one of the directories in your filesystem and type \texttt{cd <directory>}. Now run \texttt{ls -l}. What is the output?
        \end{questions}
    \end{task}

    \begin{task}
    While we are navigating the file system, we sometimes just want to go back to specific directories.
    There are three very special commands, that can be used to do this.
        \begin{questions}
            \item Type \texttt{cd}, so without any parameter and press \texttt{Enter}. Run \texttt{pwd}, where did you go?
            \item No type \texttt{cd -}, run \texttt{pwd} again, where did you go now?
            \item Run \texttt{cd ..}, run \texttt{pwd} again, what does this command do?
        \end{questions}
    \end{task}
    \begin{advice}
        If you get tired of typing the complete path when changing a directory, try tapping the \texttt{tab} key.
        It will autocomplete the path for you if possible.
        Are multiple options available and \texttt{tab} can't complete, tap the \texttt{tab} key again for a list of all possibilities!
    \end{advice}
\end{challenge}
    
